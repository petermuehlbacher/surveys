\section{Iteriertes Integral und Vertauschung der Reihenfolge}
\begin{lemma}[Reihenfolge Doppelintegral]
	Ist $f:[a,b]\times[c,d]\rightarrow \mathbb R$ stetig, dann gilt:
	
	$$\int_c^d \left(\int_a^b f(x,y) dx \right)dy = \int_a^b \left(\int_c^d f(x,y) dy \right)dx$$
\end{lemma}
\begin{proof}
	Idee: Definiere Hilfsfunktion $\varphi(y):=\int_a^b\left(\int_c^y f(x,t)dt \right)dx$, man sieht $\varphi'(y)=\int_a^b f(x,y)dx$. Setzt man nun das Integral in der verkehrten Reihenfolge an, benutzt die Identit\"at $\varphi'(y)=\int_a^b f(x,y)dx$ und $\varphi(c)=0$ erh\"alt man das gew\"unschte Ergebnis.
\end{proof}
\begin{definition}[Integral \"uber Quader]
	Sei $Q = [a_1,b_1]\times\dots\times[a_n,b_n]$ ein Quader, dann ist f\"ur jede stetige Funktion $f:Q\rightarrow \mathbb R$ das Integral von $f$ \"uber $Q$ definiert als: $\int_Q f(x)dx := \int_{a_n}^{b_n} \left(\dots\left(\int_{a_1}^{b_1}f(x_1,\dots,x_n)dx_1\right)\dots\right)dx_n$.
\end{definition}
\begin{definition}[Support einer Funktion]
	Sei $U\subseteq \mathbb R^n$, $f:U\rightarrow\mathbb R$, dann ist der Tr\"ager von $f$ definiert als: $\text{supp}f:=\overline{\{x\in U: f(x)\neq 0\}}\cap U$
\end{definition}
\begin{definition}[Vektorraum stetiger Funktionen mit kompaktem Tr\"ager]
	Mit $\mathcal C_c(U):=\{f\in \mathcal C(U) : \text{supp}f \text{ ist kompakt}\}$ bezeichnen wir den Vektorraum stetiger Funktionen mit kompaktem Tr\"ager.
\end{definition}
\begin{definition}
	F\"ur $f\in\mathcal C_c$ definieren wir $\int_{\mathbb R^n} f(x)dx := \int_Q f(x)dx$ f\"ur ein geeignetes $Q$.
\end{definition}



\section{Allgemeine Eigenschaften von Integralen}
\begin{theorem}[Eigenschaften des Integrals auf $\mathcal C_c(\mathbb R^n)$]
	Seien $f,g\in \mathcal C_c(\mathbb R^n)$, $\lambda\in \mathbb R$, dann gilt:
	
	\begin{itemize}
		\item $\int_{\mathbb R^n}(f+\lambda g)(x)dx=\int_{\mathbb R^n} f(x)dx + \lambda\int_{\mathbb R^n} g(x)dx$
		\item angenommen $\forall x\in\mathbb R^n: f(x)\leq g(x)$, dann gilt: $\int_{\mathbb R^n} f(x)dx \leq \int_{\mathbb R^n} g(x)dx$
		\item sei $a\in\mathbb R^n$, dann gilt: $\int_{\mathbb R^n} f(x+a)dx = \int_{\mathbb R^n} f(x)dx$
	\end{itemize}
\end{theorem}
\begin{proof}
	Die ersten beiden Punkte folgen direkt aus der Definition (iterierte eindimensionale Integrale), der letzte Punkt mittels Substitution in jedem der iterierten Integrale.
\end{proof}
\begin{remark}[Integral als lineares Funktional]
	Das so definierte Integral ist ein lineares, monotones und translationsinvariantes Funktional auf $\mathcal C_c(\mathbb R^n)$; man kann zeigen, dass jedes lineare Funktional mit diesen Eigenschaften (bis auf Normierung) gerade durch ein Integral gegeben sein muss.
\end{remark}

\begin{theorem}[Stetigkeit des Integrals]
	Seien $f,f_k\in\mathcal C_c(\mathbb R^n)$ f\"ur $k\in\mathbb N$ und es existiere $K$ kompakt mit $\text{supp}(f_k)\subseteq K$. Wenn $f_k \rightarrow f$ glm., dann 
	
	$$\lim_{k\rightarrow\infty}\int_{\mathbb R^n}f_k(x)dx = \int_{\mathbb R^n} f(x)dx.$$
\end{theorem}
\begin{proof}
	\todo{wof\"ur braucht man hier das $\varphi$? (im Beweis des Skriptums)}
\end{proof}





\section{Partielle Integration}
Im Folgenden bezeichne $\tilde f(x):=\begin{cases} f(x) & x\in U \\ 0 & x\in\mathbb R^n \setminus U \end{cases}$
\begin{definition}[Integral \"uber offene Teilmengen]
	F\"ur $U\subseteq \mathbb R^n$ offen, setze $\int_U f(x)dx := \int_{\mathbb R^n} \tilde f(x)dx$
\end{definition}
\begin{theorem}
	Sei $f\in\mathcal C^1(U), \varphi\in\mathcal C_c^1(U)$, dann gilt:
	$$\int_U D_j\varphi(x)=0$$
	und
	$$\int_U D_j f(x)\varphi(x)dx=-\int_U f(x) D_j\varphi(x)dx$$
\end{theorem}
\begin{proof}
	Die erste Gleichung folgt durch aufspalten in iterierte Integrale und Vertauschung der abgeleiteten Komponente nach innen; Integration \"uber $0$ bleibt $0$ und das Ergebnis folgt.
	
	Die zweite Gleichung folgt aus partieller Integration (erinnere: $D_j(f\varphi) = D_j f\varphi + f D_j\varphi$) und daraus, dass $\int_U D_j(f\varphi)=0$ ist (wegen der ersten Gleichung).
\end{proof}



\section{Prinzip von Cavalieri}
Idee: Schneide $K\subseteq \mathbb R^n$ in $(n-1)$-dimensionale ``Scheiben'' $K_t:=\{x'\in\mathbb R^{n-1}:(x',t)\in K\}\subseteq \mathbb R^{n-1}$.
\begin{theorem}[Cavalieri]
	Sei $K\subseteq \mathbb R^n$ kompakt, dann gilt:
	$$\text{Vol}_n(K) = \int_{\mathbb R}\text{Vol}_{n-1}(K_t)dt$$
\end{theorem}
\begin{proof}
	$K_t$ ist wieder kompakt und es gilt $\chi_{K_t}(x')=\chi_K(x',t)$, daher ist
	
	$$\text{Vol}_n(K) = \int_{\mathbb R}\int_{\mathbb R^{n-1}}\chi_{K_t}(x')dx'dt = \int_{\mathbb R}\text{Vol}_{n-1}(K_t)dt.$$
\end{proof}



\section{Transformationsformel (genaue Formulierung, Partition der Eins, Beweisidee)}
\begin{theorem}[Transformationsformel]
	Seien $U,V\subseteq\mathbb R^n$ offen und $\Phi : U\rightarrow V$ ein $\mathcal C^1$-Diffeomorphismus, dann gilt: $\forall f\in\mathcal C_c(V)$ ist $f\circ\Phi \in \mathcal C_c(U)$ und
	
	$$\int_U f(\Phi(x))|\text{det}D\Phi(x)|dx = \int_V f(x)dx$$
\end{theorem}
\begin{proof}
	Idee:
	\begin{itemize}
		\item Jeder Diffeomorphismus $G$ l\"a\ss t sich lokal als Verknüpfung zweier Typen aufspalten: \\Typ A: $G(x)=(x_1,\dots,x_{m-1},g(x),x_{m+1},\dots,x_n)$\\Typ B: Transposition die zwei Basisvektoren vertauscht und alle anderen fix l\"a\ss t
		\item Die Partition der Eins erm\"oglicht genau so eine Lokalisierung.
		\item Durch Aufspalten in iterierte Integrale kann man die Substitutionsregel aus dem Eindimensionalen auf Typ A anwenden, Typ B kehrt lediglich das Vorzeichen um.
	\end{itemize}
	\todo{diesen Beweis nochmal genau anschauen}
\end{proof}


\section{Oberfl\"achenintegrale \"uber Funktionen und \"uber Vektorfelder}
\begin{definition}[Oberfl\"achenintegral im $\mathbb R^3$]
	Sei $T\subseteq \mathbb R^2$ offen, $\Phi:T\rightarrow \mathbb R^3$ eine Immersion und $K\subseteq T$ kompakt, dann definiert man f\"ur stetige $f:\Phi(K)\rightarrow \mathbb R$ das \textit{Oberfl\"achenintegral}
	$$\int_{\Phi(K)}fdS := \int_K f(\Phi(t))||N_\Phi(t)||dt$$ wobei $N_\Phi(t) = D_{t_1}\Phi(t)\times D_{t_2}\Phi(t)$, also der Normalenvektor (=Kreuzprodukt der Tangentialvektoren) ist.
\end{definition}

\begin{lemma}[Unabh\"angigkeit von der Parametrisierung]
	
\end{lemma}






\section{Lebesgue-Integral}
Siehe \href{http://homepage.univie.ac.at/karlheinz.groechenig/lebesgue.pdf}{Einleitung von Gr\"ochenig} (ist bereits kurz und b\"undig zusammengefasst).
